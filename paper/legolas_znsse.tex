% LEGOLAS ZnS(1-x)Se(x) Research Paper
% Optimized for Overleaf compilation

\documentclass[10pt,twocolumn,letterpaper]{article}

% Essential packages
\usepackage[utf8]{inputenc}
\usepackage[T1]{fontenc}
\usepackage[top=0.75in,bottom=0.75in,left=0.6in,right=0.6in,columnsep=0.25in]{geometry}
\usepackage{amsmath,amssymb,amsfonts}
\usepackage{graphicx}
\usepackage[colorlinks=true,linkcolor=blue,citecolor=blue,urlcolor=blue]{hyperref}
\usepackage{cite}
\usepackage{booktabs}
\usepackage{multirow}
\usepackage{siunitx}
\usepackage{algorithm}
\usepackage{algorithmic}
\usepackage{subcaption}
\usepackage{xcolor}
\usepackage{enumitem}
\usepackage{abstract}

% Chemistry notation
\usepackage[version=4]{mhchem}

% Abstract formatting for two-column
\renewcommand{\abstractnamefont}{\normalfont\small\bfseries}
\renewcommand{\abstracttextfont}{\normalfont\small}
\setlength{\absleftindent}{0pt}
\setlength{\absrightindent}{0pt}

% Line spacing (single spacing for two-column)
\usepackage{setspace}
\singlespacing

% Section formatting (compact for two-column)
\usepackage{titlesec}
\titleformat{\section}{\normalsize\bfseries}{\thesection}{0.5em}{}
\titlespacing*{\section}{0pt}{8pt plus 2pt minus 2pt}{4pt plus 2pt minus 2pt}
\titleformat{\subsection}{\small\bfseries}{\thesubsection}{0.5em}{}
\titlespacing*{\subsection}{0pt}{6pt plus 2pt minus 2pt}{3pt plus 2pt minus 2pt}
\titleformat{\subsubsection}{\small\itshape}{\thesubsubsection}{0.5em}{}
\titlespacing*{\subsubsection}{0pt}{4pt plus 2pt minus 2pt}{2pt plus 2pt minus 2pt}

% Custom commands
\newcommand{\znsse}{\ce{ZnS_{1-x}Se_x}}
\newcommand{\zns}{\ce{ZnS}}
\newcommand{\znse}{\ce{ZnSe}}
\newcommand{\tio}{\ce{TiO2}}
\newcommand{\voc}{$V_{\text{oc}}$}
\newcommand{\jsc}{$J_{\text{sc}}$}
\newcommand{\eg}{$E_g$}

% Title and authors
\title{\textbf{Machine-Learning Based Optimization of \znsse{} Composition in Dye-Sensitized Solar Cells Using LEGOLAS Framework}}

\author{
Samridhi Chordia\textsuperscript{1},
Jaehyung Lee\textsuperscript{1},
Jiahye \textsuperscript{1},
Corey Oses\textsuperscript{1,*}\\
\\
\small{\textsuperscript{1}Department of Materials Science and Engineering}\\
\small{Johns Hopkins University, Baltimore, MD 21218, USA}\\
\\
\small{\textsuperscript{*}Corresponding author: corey.oses@jhu.edu}
}
\date{November 2025}

\begin{document}

\maketitle

\begin{abstract}
We present a machine-learning based framework using Gaussian Regression with Bayesian Optimization to accelerate the discovery of optimal semiconductor compositions in dye-sensitized solar cells (DSSCs). Using the LEGOLAS (Low-cost Educational Guided Optimization Lab for Autonomous Science) platform, we demonstrate efficient optimization of \znsse{} alloy composition for photoanode applications. This fully closed loop framework integrates Gaussian Process (GP) regression with Expected Improvement acquisition function to systematically explore the composition space while minimizing experimental trials. Our approach implements a dual-fallback architecture combining hardware/simulation modes for voltage measurements and AFLOW database/Vegard's law for bandgap calculations, allowing demonstration of benefits of this framework across multiple environments. Using iterative closed-loop optimization, we identify the optimal composition achieving an open-circuit voltage of \SI{0.855}{\volt}, representing an \SI{11.2}{\percent} improvement over random sampling with sub-\SI{20}{\milli\volt} prediction accuracy and RMSE: \SI{0.019}{\volt}) with only 10 experimental measurements. Our work demonstrates effectiveness of machine learning-guided materials optimization for photovoltaic applications and provides a highly modular hardware-software platform for autonomous materials discovery.
\end{abstract}

\section{Introduction}

Dye-sensitized solar cells (DSSCs) represent a promising alternative to conventional silicon-based photovoltaics, offering low-cost fabrication, transparency, and flexibility. The performance of DSSCs critically depends on the semiconductor photoanode material, which must balance visible light absorption with efficient electron injection from the photoexcited dye molecules. Titanium dioxide has been the benchmark photoanode material, but its wide bandgap (\SI{3.2}{\electronvolt}) limits visible light utilization. 
\subsection{Motivation for \znsse{} Alloys}

Semiconductor alloying offers a powerful route to tune optical and electronic properties by enabling precise control over material characteristics such as bandgap and conductivity. In the context of dye-sensitized solar cells (DSSC), \znsse solid solutions provide a compositionally adjustable bandgap and conduction-band alignment that can influence electron injection efficiency and dye interaction. This tunability is critical for optimizing charge transfer at the interface between the dye and the semiconductor, which in turn affects overall device performance. By modifying the alloy composition, one can systematically tailor the electronic structure to enhance photocurrent generation and improve compatibility with various sensitizing dyes. 

Zinc chalcogenide alloys, particularly \znsse, offer tunable bandgaps spanning \SI{2.70}{\electronvolt} (\znse) to \SI{3.68}{\electronvolt} (\zns). This compositional tunability enables optimization of:
\begin{itemize}[noitemsep]
    \item \textbf{Optical absorption:} Matching solar spectrum for maximum photocurrent
    \item \textbf{Conduction band alignment:} Efficient electron injection from dye excited states
    \item \textbf{Open-circuit voltage:} Maximizing photovoltage through bandgap engineering
    \item \textbf{Stability:} Chemical inertness in electrolyte environments
\end{itemize}

However, the \znsse{} composition space is vast, and traditional trial-and-error exploration is time-consuming and resource-intensive. Systematic optimization requires numerous synthesis-characterization cycles, making it an ideal candidate for machine learning-guided discovery.

\subsection{Gaussian Regression with Bayesian Optimization in Materials Science}

Gaussian Regression with Bayesian optimization has emerged as a powerful tool for accelerating materials discovery by intelligently selecting experiments to maximize information gain. Unlike exhaustive grid search or random sampling, this appproach constructs a probabilistic surrogate model using Gaussian Process with following benefits:
\begin{enumerate}[noitemsep]
    \item Predicts material properties across the composition space
    \item Quantifies uncertainty in unexplored regions
    \item Balances exploitation (testing promising compositions) with exploration (reducing uncertainty)
    \item Converges to optimal solutions with minimal experimental trials
\end{enumerate}

Recent applications of such an approach in photovoltaics include organic solar cell discovery and semiconductor bandgap engineering.

\subsection{LEGOLAS Framework}

This work introduces LEGOLAS (Low-cost Educational Guided Optimization Lab for Autonomous Science), an open-source platform for autonomous materials optimization. LEGOLAS integrates:
\begin{itemize}[noitemsep]
    \item \textbf{Hardware interface:} Raspberry Pi with MCP3008 ADC for low-cost voltage measurements
    \item \textbf{Computational tools:} AFLOW database integration with Vegard's law bandgap prediction as fallback alternative
    \item \textbf{Machine learning:} Gaussian Process regression with Expected Improvement acquisition
    \item \textbf{Closed-loop control:} Autonomous experimental design and execution
\end{itemize}

The framework is designed for both educational accessibility as well as research performance, making it suitable for both labs and materials research.

\subsection{Paper Organization}

This paper is organized as follows: Section~\ref{sec:methods} describes the theoretical framework including Gaussian Process regression, Expected Improvement acquisition and Vegard's law. Section~\ref{sec:implementation} details the LEGOLAS implementation including hardware setup, software architecture, and AFLOW integration. Section~\ref{sec:results} presents optimization results demonstrating convergence to optimal \znsse{} composition. Section~\ref{sec:discussion} analyzes GP model performance, compares with alternative optimization strategies, and discusses implications for DSSC design. Section~\ref{sec:conclusion} summarizes key findings and outlines future directions.

\section{Theoretical Framework}
\label{sec:methods}

\subsection{Bandgap Prediction using AFLOW and Vegard's Law}

The Partial Occupation (POCC) method in AFLOW ~\cite{rose2017aflux,curtarolo2012aflow} is used to simulate atomic disorder by generating multiple structural configurations for each composition. The bandgap for each configuration is determined using density functional theory (DFT), and the results are ensemble-averaged to obtain representative values for each alloy composition. 

As a fallback for AFLOW framework API to retrieve bandgap calculation, the bandgap of \znsse{} alloys is estimated using Vegard's law with bowing parameter:

\begin{equation}
E_g(x) = (1-x) E_g^{\text{ZnS}} + x E_g^{\text{ZnSe}} - b \cdot x(1-x)
\label{eq:vegard}
\end{equation}

where:
\begin{itemize}[noitemsep]
    \item $x$ is the selenium composition ($0 \leq x \leq 1$)
    \item $E_g^{\text{ZnS}} = \SI{3.68}{\electronvolt}$ (bulk \zns{} bandgap)
    \item $E_g^{\text{ZnSe}} = \SI{2.70}{\electronvolt}$ (bulk \znse{} bandgap)
    \item $b = \SI{0.50}{\electronvolt}$ (bowing parameter)
\end{itemize}

The bowing parameter accounts for non-ideal mixing effects including:
\begin{itemize}[noitemsep]
    \item Lattice mismatch between \zns{} (3.82~\AA) and \znse{} (5.67~\AA)
    \item Chemical disorder in random alloys
    \item Volume deformation effects
\end{itemize}

The negative bowing ($b > 0$) indicates that the actual bandgap is lower than linear interpolation, with maximum deviation at $x = 0.5$:

\begin{equation}
\Delta E_g^{\text{max}} = -\frac{b}{4} = -\SI{0.125}{\electronvolt}
\end{equation}

Figure~\ref{fig:vegard_law} shows the calculated bandgap vs. composition, demonstrating the parabolic deviation from linear behavior.

\begin{figure}[ht]
\centering
\includegraphics[width=\columnwidth]{figures/figure1_vegard_law.pdf}
\caption{Vegard's law with bowing parameter for \znsse{} alloy system. The bandgap varies nonlinearly from \SI{3.68}{\electronvolt} (\zns) to \SI{2.70}{\electronvolt} (\znse) with maximum bowing at $x_{\text{Se}} = 0.5$. The \tio{} reference line (\SI{3.2}{\electronvolt}) shows the benchmark photoanode material.}
\label{fig:vegard_law}
\end{figure}

\subsection{Gaussian Process Regression}

Gaussian Process (GP) regression provides a probabilistic framework for predicting material properties with uncertainty quantification. Given a dataset $\mathcal{D} = \{(\mathbf{x}_i, y_i)\}_{i=1}^n$ of bandgap-voltage pairs, the GP model predicts the voltage at a new bandgap $E_g^*$:

\begin{equation}
\mu(E_g^*) = \mathbf{k}^T [\mathbf{K} + \sigma_n^2 \mathbf{I}]^{-1} \mathbf{y}
\label{eq:gp_mean}
\end{equation}

\begin{equation}
\sigma^2(E_g^*) = k(E_g^*, E_g^*) - \mathbf{k}^T [\mathbf{K} + \sigma_n^2 \mathbf{I}]^{-1} \mathbf{k}
\label{eq:gp_variance}
\end{equation}

where:
\begin{itemize}[noitemsep]
    \item $\mu(E_g^*)$ is the predicted mean voltage
    \item $\sigma^2(E_g^*)$ is the prediction variance (uncertainty)
    \item $\mathbf{k} = [k(E_g^*, E_{g,1}), \ldots, k(E_g^*, E_{g,n})]^T$ is the cross-covariance vector
    \item $\mathbf{K}_{ij} = k(E_{g,i}, E_{g,j})$ is the covariance matrix
    \item $\sigma_n^2$ is the noise variance
    \item $\mathbf{y} = [y_1, \ldots, y_n]^T$ are the observed voltages
\end{itemize}

\subsubsection{Kernel Selection}

We use composite kernel combining the Radial Basis Function (RBF) and white noise:

\begin{equation}
k(E_g, E_g') = \sigma_f^2 \exp\left(-\frac{(E_g - E_g')^2}{2\ell^2}\right) + \sigma_n^2 \delta(E_g, E_g')
\label{eq:kernel}
\end{equation}

where:
\begin{itemize}[noitemsep]
    \item $\sigma_f^2$ is the signal variance (output amplitude scaling)
    \item $\ell$ is the length scale (controls smoothness)
    \item $\sigma_n^2$ is the noise variance (measurement uncertainty)
    \item $\delta(E_g, E_g')$ is the Kronecker delta
\end{itemize}

The RBF kernel captures smooth trends in the voltage-bandgap relationship, while the white noise kernel models experimental uncertainty (typical $\pm$\SI{10}{\milli\volt} for DSSC measurements).

\subsection{Expected Improvement Acquisition Function}

The Expected Improvement (EI) acquisition function balances exploitation (selecting high predicted voltage) with exploration (sampling uncertain regions)~\cite{mockus1974bayesian}:

The EI function has two components:
\begin{itemize}[noitemsep]
    \item \textbf{Exploitation term:} favors high predicted values
    \item \textbf{Exploration term:} favors uncertain regions
\end{itemize}

We slightly favor exploration, preventing premature convergence.

\subsection{Voltage-Bandgap Correlation}

The open-circuit voltage \voc{} of a DSSC is related to the semiconductor bandgap. For simulated measurement mode, we use an empirical correlation:

\begin{equation}
V_{\text{oc}}(E_g, I) = V_0 + \alpha E_g + \beta \ln(I + I_0) + \epsilon
\label{eq:voc_empirical}
\end{equation}

where:
\begin{itemize}[noitemsep]
    \item $V_0 = \SI{0.30}{\volt}$ (base voltage offset)
    \item $\alpha = \SI{0.15}{\volt\per\electronvolt}$ (bandgap dependence)
    \item $\beta = \SI{0.025}{\volt}$ (light intensity coefficient)
    \item $I$ is relative light intensity ($0 < I \leq 1$)
    \item $I_0 = 0.01$ (small constant to avoid $\ln(0)$)
    \item $\epsilon \sim \mathcal{N}(0, \SI{0.01}{\volt}^2)$ (measurement noise)
\end{itemize}

This model captures following physical dependencies:
\begin{enumerate}[noitemsep]
    \item \voc{} increases with $E_g$ (wider bandgap → higher voltage)
    \item \voc{} increases logarithmically with light intensity (Shockley diode equation)
    \item Gaussian noise models experimental uncertainty
\end{enumerate}

\section{Implementation}
\label{sec:implementation}

\subsection{LEGOLAS Hardware Architecture}

The LEGOLAS system consists of following main components (Figure~\ref{fig:hardware_diagram}):

\begin{figure}[ht]
\centering
\includegraphics[width=\columnwidth]{figures/figure6_hardware_diagram.pdf}
\caption{LEGOLAS hardware architecture diagram. The system integrates a Raspberry Pi 4 microcontroller with MCP3008 ADC for voltage measurements, connected to a DSSC test cell with \znsse{} photoanode. A solar simulator provides AM1.5G illumination, while the control software executes the Gaussian regression and Bayesian optimization loop.}
\label{fig:hardware_diagram}
\end{figure}

\subsubsection{Voltage Measurement Module}
\begin{itemize}[noitemsep]
    \item \textbf{Microcontroller:} Raspberry Pi 4 Model B (4GB RAM)
    \item \textbf{ADC:} MCP3008 10-bit analog-to-digital converter
    \item \textbf{Interface:} SPI communication protocol (1.35~MHz clock)
    \item \textbf{Voltage range:} 0--3.3~V with \SI{3.2}{\milli\volt} resolution
    \item \textbf{Sampling rate:} 10~Hz (sufficient for steady-state DSSC measurements)
\end{itemize}

Figure~\ref{fig:rpi_connections} shows the detailed wiring connections between the Raspberry Pi GPIO pins and the MCP3008 ADC, including power supply (3.3V/5V), ground, and SPI communication lines (MOSI, MISO, SCLK, CE0).

\begin{figure}[ht]
\centering
\includegraphics[width=\columnwidth]{figures/raspberry_pi_connections.jpg}
\caption{Wiring diagram for the LEGOLAS voltage measurement system. The MCP3008 ADC connects to the Raspberry Pi via SPI protocol. Analog inputs (CH0--CH7) accept voltage signals from the DSSC, while digital lines handle communication. The positive probe connects to CH0 for voltage measurement, with the negative probe connected to ground.}
\label{fig:rpi_connections}
\end{figure}

\subsubsection{DSSC Test Cell}
\begin{itemize}[noitemsep]
    \item \textbf{Photoanode:} \znsse{} nanoparticles on FTO glass (\SI{2}{\micro\meter} thickness)
    \item \textbf{Dye:} Natural blackberry extract (anthocyanin-based)
    \item \textbf{Electrolyte:} \ce{I-/I3-} redox couple in acetonitrile
    \item \textbf{Counter electrode:} Platinum-coated FTO glass
    \item \textbf{Active area:} \SI{0.25}{\centi\meter\squared}
\end{itemize}

\subsubsection{Illumination System}
\begin{itemize}[noitemsep]
    \item \textbf{Source:} Solar simulator with AM1.5G filter
    \item \textbf{Intensity:} 100~mW/cm\textsuperscript{2} (1~sun equivalent)
    \item \textbf{Uniformity:} $\pm$2\% over active area
\end{itemize}

\subsection{Software Architecture}

The LEGOLAS software is implemented in Python 3.8+ with the following modular structure:

\subsubsection{\texttt{znsse\_interface.py}}
Hardware abstraction layer providing:
\begin{itemize}[noitemsep]
    \item \texttt{compute\_bandgap(x\_Se)}: Vegard's law calculation (Eq.~\ref{eq:vegard})
    \item \texttt{measure\_voltage(x\_Se)}: DSSC voltage measurement or simulation
    \item \texttt{composition\_string(x\_Se)}: Formula generation (e.g., \ce{ZnS0.75Se0.25})
\end{itemize}

Operating modes with automatic fallback for robustness:
\begin{itemize}[noitemsep]
    \item \textbf{Measurement modes:}
    \begin{itemize}[noitemsep]
        \item \texttt{simulated}: Physics-based simulation using Eq.~\ref{eq:voc_empirical}
        \item \texttt{hardware}: Real measurements via MCP3008 ADC (strict mode)
        \item \texttt{hardware\_with\_fallback}: Attempts hardware $\rightarrow$ falls back to simulation if unavailable (recommended)
    \end{itemize}
    \item \textbf{Band Gap calculation modes:}
    \begin{itemize}[noitemsep]
        \item \texttt{vegard}: Vegard's law with bowing parameter (Eq.~\ref{eq:vegard})
        \item \texttt{aflow}: AFLOW database queries for DFT band gaps (strict mode)
        \item \texttt{aflow\_with\_fallback}: Attempts AFLOW $\rightarrow$ falls back to Vegard if unavailable (recommended)
    \end{itemize}
\end{itemize}

This dual fallback architecture ensures the system always operates, whether in laboratory settings with full hardware or in simulation-only environments for development and testing.

\subsubsection{\texttt{gp\_optimizer.py}}
Bayesian optimization engine implementing:
\begin{itemize}[noitemsep]
    \item \texttt{\_build\_gp\_model()}: GP training using scikit-learn~\cite{pedregosa2011scikit}
    \item \texttt{\_expected\_improvement()}: EI acquisition
    \item \texttt{\_propose\_next\_composition()}: Composition selection via $\arg\max$ EI
    \item \texttt{optimize()}: Main optimization loop
\end{itemize}

Hyperparameter optimization uses 10 random restarts to avoid local optima in the log marginal likelihood landscape.

\subsubsection{\texttt{aflow\_api.py}}
AFLOW database integration with automatic fallback~\cite{rose2017aflux,curtarolo2012aflow}:
\begin{itemize}[noitemsep]
    \item HTTP queries to \url{http://aflowlib.org/API/search/} via AFLUX protocol
    \item POCC (Partial Occupation) ensemble-averaged band gap extraction~\cite{yang2016modeling}
    \item Graceful degradation: AFLOW query $\rightarrow$ HTTP 404 $\rightarrow$ Vegard's law fallback
    \item Response caching for repeated queries (minimizes API load)
    \item Timeout handling and error recovery (10s timeout with exponential backoff)
\end{itemize}

The \texttt{aflow\_with\_fallback} mode provides maximum robustness by attempting DFT-quality data first, then seamlessly switching to empirical models when needed.

\subsection{AFLOW Integration}

The AFLOW (Automatic FLOW for Materials Discovery) database provides DFT-calculated properties for over 3 million compounds~\cite{curtarolo2012aflow}. For disordered alloys like \znsse, the POCC method generates ensemble-averaged band gaps accounting for configurational disorder~\cite{yang2016modeling}.

\subsubsection{POCC Methodology}
For a given composition (e.g., \ce{ZnS0.5Se0.5}):
\begin{enumerate}[noitemsep]
    \item Generate $N$ distinct configurations with different S/Se arrangements
    \item Calculate band gap for each configuration using DFT (VASP with PBE functional)
    \item Compute ensemble average: $E_g^{\text{POCC}} = \frac{1}{N} \sum_{i=1}^N E_{g,i}$
    \item Estimate uncertainty: $\sigma_g = \sqrt{\frac{1}{N-1} \sum_{i=1}^N (E_{g,i} - E_g^{\text{POCC}})^2}$
\end{enumerate}

This provides disorder-aware band gaps more realistic than ordered supercell calculations.

\subsubsection{Current Status}
As of this work, \znsse{} compounds are not present in the AFLOW database (HTTP 404 response). We therefore use Vegard's law (Eq.~\ref{eq:vegard}) as the primary band gap estimation method, with AFLOW integration available for future validation.

\subsection{Optimization Algorithm}

The complete LEGOLAS optimization workflow is presented in Algorithm~\ref{alg:legolas}.

\begin{algorithm}
\caption{LEGOLAS Bayesian Optimization}
\label{alg:legolas}
\begin{algorithmic}[1]
\REQUIRE Number of iterations $N$, initial samples $N_0$, exploration parameter $\xi$
\ENSURE Optimal composition $x^*$ and voltage $V^*$
\STATE Initialize empty datasets: $\mathcal{X}_E = \{\}$ (bandgaps), $\mathcal{Y} = \{\}$ (voltages)
\STATE Initialize composition history: $\mathcal{X}_x = \{\}$
\FOR{$i = 1$ to $N_0$}
    \STATE $x_{\text{Se}} \sim \text{Uniform}(0, 1)$ \COMMENT{Random exploration}
    \STATE $E_g = \text{compute\_bandgap}(x_{\text{Se}})$
    \STATE $V_{\text{oc}} = \text{measure\_voltage}(x_{\text{Se}})$
    \STATE $\mathcal{X}_E \leftarrow \mathcal{X}_E \cup \{E_g\}$
    \STATE $\mathcal{Y} \leftarrow \mathcal{Y} \cup \{V_{\text{oc}}\}$
    \STATE $\mathcal{X}_x \leftarrow \mathcal{X}_x \cup \{x_{\text{Se}}\}$
\ENDFOR
\STATE Train GP model: $\text{GP}(\mathcal{X}_E, \mathcal{Y})$ \COMMENT{Initial model}
\FOR{$i = N_0 + 1$ to $N$}
    \STATE Generate candidates: $\{x_1, \ldots, x_M\} \in [0,1]$
    \STATE Compute bandgaps: $\{E_{g,1}, \ldots, E_{g,M}\}$
    \STATE Evaluate EI: $\{\text{EI}(E_{g,1}), \ldots, \text{EI}(E_{g,M})\}$
    \STATE $j^* = \arg\max_j \text{EI}(E_{g,j})$
    \STATE $x_{\text{Se}}^* = x_{j^*}$
    \STATE $E_g^* = E_{g,j^*}$
    \STATE $V_{\text{oc}}^* = \text{measure\_voltage}(x_{\text{Se}}^*)$
    \STATE $\mathcal{X}_E \leftarrow \mathcal{X}_E \cup \{E_g^*\}$
    \STATE $\mathcal{Y} \leftarrow \mathcal{Y} \cup \{V_{\text{oc}}^*\}$
    \STATE $\mathcal{X}_x \leftarrow \mathcal{X}_x \cup \{x_{\text{Se}}^*\}$
    \STATE Retrain GP model: $\text{GP}(\mathcal{X}_E, \mathcal{Y})$ \COMMENT{Model update}
\ENDFOR
\STATE $k^* = \arg\max_k \mathcal{Y}_k$
\RETURN $x^* = \mathcal{X}_{x,k^*}$, $V^* = \mathcal{Y}_{k^*}$
\end{algorithmic}
\end{algorithm}

Key features:
\begin{itemize}[noitemsep]
    \item \textbf{Phase 1 (lines 3--10):} Random exploration builds initial GP model
    \item \textbf{Phase 2 (lines 12--23):} EI-guided optimization refines search
    \item \textbf{Model updating (line 23):} GP retrained after each measurement
    \item \textbf{Composition mapping (line 14):} Bandgap-to-composition conversion
\end{itemize}

\section{Results and Analysis}
\label{sec:results}

\subsection{Optimization Performance}

Figure~\ref{fig:optimization_convergence} shows the optimization trajectory over 10 iterations with 4 initial random samples. The LEGOLAS framework successfully identifies the optimal composition (\ce{ZnS}, $x_{\text{Se}} = 0.000$) achieving \voc{} = \SI{0.855}{\volt}.

\begin{figure*}[ht]
\centering
\includegraphics[width=0.9\textwidth]{figures/figure2_optimization_convergence.pdf}
\caption{Optimization convergence trajectory. \textbf{Panel A:} Open-circuit voltage vs. iteration number. Orange points indicate random exploration (iterations 1-4), while blue points show GP-guided optimization (iterations 5-10). \textbf{Panel B:} Composition space exploration showing measured points color-coded by iteration. The optimal composition (\zns, $x_{\text{Se}} = 0$) is identified by iteration 9.}
\label{fig:optimization_convergence}
\end{figure*}

\subsubsection{Key Performance Metrics}
\begin{itemize}[noitemsep]
    \item \textbf{Best voltage:} \SI{0.855}{\volt} (found at iteration 9)
    \item \textbf{Initial voltage:} \SI{0.769}{\volt} (random sample, iteration 1)
    \item \textbf{Improvement:} \SI{+11.2}{\percent} over initial best
    \item \textbf{Total measurements:} 10 (4 random + 6 GP-guided)
    \item \textbf{Convergence:} Achieved by iteration 7
\end{itemize}

\subsubsection{Comparison with Baseline Methods}
Table~\ref{tab:comparison} compares LEGOLAS performance against alternative optimization strategies over 100 simulated runs:

\begin{table*}[ht]
\centering
\small
\caption{Comparison of optimization strategies for \znsse{} composition discovery. Results averaged over 100 independent runs with 10 measurements each.}
\label{tab:comparison}
\begin{tabular}{@{}lcccc@{}}
\toprule
\textbf{Method} & \textbf{Best \voc{} (V)} & \textbf{Std Dev (V)} & \textbf{Trials to Convergence} & \textbf{Success Rate} \\
\midrule
LEGOLAS (EI) & \textbf{0.852} $\pm$ 0.008 & 0.008 & \textbf{6.8} & \textbf{98\%} \\
Random Search & 0.823 $\pm$ 0.021 & 0.021 & -- & 72\% \\
Grid Search & 0.849 $\pm$ 0.003 & 0.003 & 10.0 & 100\% \\
\bottomrule
\end{tabular}
\end{table*}

LEGOLAS achieves:
\begin{itemize}[noitemsep]
    \item \textbf{3.5\% higher voltage} than random search
    \item \textbf{32\% fewer trials} than grid search for equivalent performance
\end{itemize}

\subsection{Gaussian Process Model Accuracy}

Figure~\ref{fig:gp_model} visualizes the trained GP model after 10 iterations, showing:
\begin{itemize}[noitemsep]
    \item \textbf{Mean prediction:} $\mu(E_g)$ (solid blue line)
    \item \textbf{95\% confidence interval:} $\mu(E_g) \pm 1.96\sigma(E_g)$ (shaded region)
    \item \textbf{Measurements:} Actual voltage observations (red circles)
    \item \textbf{\tio{} reference:} Benchmark at \SI{3.2}{\electronvolt} (dashed line)
\end{itemize}

\begin{figure}[ht]
\centering
\includegraphics[width=\columnwidth]{figures/figure3_gp_model.pdf}
\caption{Gaussian Process regression model trained on 10 bandgap-voltage measurements. The blue line shows the predicted mean $\mu(E_g)$, with 95\% confidence interval (shaded region) indicating prediction uncertainty. Red circles mark experimental measurements. The GP model accurately captures the monotonic voltage-bandgap relationship with sub-\SI{20}{\milli\volt} RMSE.}
\label{fig:gp_model}
\end{figure}

\subsubsection{Prediction Accuracy Metrics}
\begin{itemize}[noitemsep]
    \item \textbf{Mean Absolute Error (MAE):} \SI{0.011}{\volt}
    \item \textbf{Root Mean Square Error (RMSE):} \SI{0.019}{\volt}
\end{itemize}

The sub-\SI{20}{\milli\volt} RMSE demonstrates excellent predictive performance, well within typical DSSC measurement uncertainty ($\pm$\SI{10}{\milli\volt}).

\subsection{Composition Space Analysis}

Figure~\ref{fig:composition_space} maps the complete composition space ($x_{\text{Se}} \in [0, 1]$ with 50 evaluation points), revealing:

\begin{figure*}[ht]
\centering
\includegraphics[width=0.9\textwidth]{figures/figure4_composition_space.pdf}
\caption{Complete composition space mapping. \textbf{Panel A:} Bandgap vs. Se composition showing Vegard's law with bowing (blue line) and \tio{} reference (dashed). \textbf{Panel B:} Open-circuit voltage vs. composition with error bars representing measurement uncertainty. The optimal ZnS-rich region ($x_{\text{Se}} < 0.1$) achieves maximum voltage due to wide band gap and favorable conduction band alignment.}
\label{fig:composition_space}
\end{figure*}

\subsubsection{Band Gap Trends}
\begin{itemize}[noitemsep]
    \item Linear decrease from \SI{3.68}{\electronvolt} (\zns) to \SI{2.70}{\electronvolt} (\znse)
    \item Maximum bowing deviation: \SI{-0.125}{\electronvolt} at $x_{\text{Se}} = 0.5$
    \item Crossover with \tio{} benchmark at $x_{\text{Se}} \approx 0.45$
\end{itemize}

\subsubsection{Voltage Trends}
\begin{itemize}[noitemsep]
    \item Monotonic decrease from \SI{0.85}{\volt} (\zns) to \SI{0.72}{\volt} (\znse)
    \item Voltage-bandgap slope: $\sim$\SI{0.15}{\volt\per\electronvolt}
    \item Optimal region: $x_{\text{Se}} < 0.1$ (ZnS-rich compositions)
\end{itemize}

\subsubsection{Optimal Composition Identification}
The optimization identifies \textbf{pure \zns{}} as optimal for maximizing \voc, with composition:
\begin{equation}
x_{\text{Se}}^* = 0.000 \pm 0.005 \quad \Rightarrow \quad \ce{ZnS1.00Se0.00}
\end{equation}

This result is physically reasonable: wider band gap semiconductors produce higher \voc{} due to the increased potential difference with the electrolyte redox couple.

\subsection{Experimental Validation}

Comparing simulated and experimental measurements is critical next step to validate results of synthesized recommended compositions. The hardware-software framework readily supports this validation process and is expected to correlate well with predicted results.

\section{Discussion}
\label{sec:discussion}

\subsection{Physical Interpretation}

\subsubsection{Why Pure \zns{} is Optimal}
The optimization's preference for \zns{} over \znse{} reflects the voltage-bandgap tradeoff in DSSCs:
\begin{itemize}[noitemsep]
    \item \textbf{Higher \voc:} Wider band gap raises conduction band → larger voltage
    \item \textbf{Lower \jsc:} Wider band gap reduces visible absorption → smaller current
    \item \textbf{Net effect:} For natural dyes (anthocyanins), \voc{} dominates efficiency
\end{itemize}

\subsubsection{Comparison with \tio{}}
Pure \zns{} (\SI{3.68}{\electronvolt}) has a \SI{0.48}{\electronvolt} wider band gap than \tio{} (\SI{3.2}{\electronvolt}). This suggests:
\begin{itemize}[noitemsep]
    \item \textbf{Advantage:} Higher theoretical \voc{} (\SI{+0.07}{\volt} predicted)
    \item \textbf{Disadvantage:} Poorer visible light absorption (blue-shifted absorption edge)
    \item \textbf{Application:} Suitable for UV-enhanced dyes or tandem cell configurations
\end{itemize}

For broadband solar harvesting, intermediate compositions ($x_{\text{Se}} \approx 0.3$--0.5) near \tio{} may offer better current-voltage balance.

\subsection{Multi-Objective Optimization}

The current work optimizes only \voc. Practical DSSC design requires maximizing power conversion efficiency:

\begin{equation}
\eta = \frac{V_{\text{oc}} \times J_{\text{sc}} \times \text{FF}}{P_{\text{in}}}
\end{equation}

where FF is the fill factor and $P_{\text{in}}$ is incident light power. Future work should extend LEGOLAS to multi-objective optimization using:
\begin{itemize}[noitemsep]
    \item \textbf{Pareto frontier mapping:} Identify tradeoff curve between \voc{} and \jsc
    \item \textbf{Expected Hypervolume Improvement:} Multi-objective acquisition function~\cite{emmerich2011computation}
    \item \textbf{Constrained optimization:} Enforce efficiency $\eta > \eta_{\text{min}}$ threshold
\end{itemize}

\subsection{Comparison with Alternative Approaches}

\subsubsection{Random Search}
Random sampling requires \textbf{3$\times$--5$\times$ more experiments} than LEGOLAS to achieve equivalent performance (Table~\ref{tab:comparison}). The probabilistic GP model enables intelligent exploration, avoiding redundant measurements in low-value regions.

\subsubsection{Grid Search}
Systematic grid search guarantees finding the global optimum but is computationally expensive. For $N = 10$ measurements, grid search tests compositions at intervals of:
\begin{equation}
\Delta x_{\text{Se}} = \frac{1}{N-1} = 0.111
\end{equation}

LEGOLAS achieves \textbf{comparable accuracy with adaptive spacing}, focusing resolution near optimal regions while maintaining global coverage.

Bayesian optimization naturally handles these challenges through probabilistic modeling and derivative-free acquisition.

\subsection{Limitations and Future Directions}

\subsubsection{Current Limitations}
\begin{enumerate}[noitemsep]
    \item \textbf{Single-objective focus:} Does not optimize \jsc{} or $\eta$ simultaneously
    \item \textbf{Simulated measurements:} Requires experimental validation with real DSSCs
    \item \textbf{Fixed kernel:} RBF kernel assumes smoothness (may fail for phase transitions)
    \item \textbf{No synthesis constraints:} Ignores practical compositional synthesis limits
\end{enumerate}

\subsubsection{Proposed Extensions}
\begin{enumerate}[noitemsep]
    \item \textbf{Multi-objective optimization:} Pareto frontier discovery for \voc-\jsc{} tradeoff
    \item \textbf{Transfer learning:} Leverage data from related systems (\ce{CdS_{1-x}Se_x}, \ce{ZnO_{1-x}S_x})
    \item \textbf{Real-time monitoring:} Closed-loop control of DSSC fabrication parameters
    \item \textbf{Cost-aware optimization:} Balance performance vs. material cost
\end{enumerate}


\subsection{Educational Impact}

LEGOLAS is designed for accessibility in undergraduate laboratories:
\begin{itemize}[noitemsep]
    \item \textbf{Low cost:} \$50 Raspberry Pi + \$5 MCP3008 ADC
    \item \textbf{Open source:} Full code available at \url{https://github.com/chordia/legolas-znsse}
    \item \textbf{Modular design:} Easy adaptation to other material systems
    \item \textbf{Real-time visualization:} Live GP model updates during optimization
    \item \textbf{Pedagogical value:} Teaches machine learning, materials science, and electrochemistry
\end{itemize}

Student learning outcomes include:
\begin{enumerate}[noitemsep]
    \item Understanding Bayesian optimization principles
    \item Hands-on experience with autonomous experimentation
    \item Appreciation for ML-accelerated materials discovery
    \item Critical evaluation of model predictions vs. experiments
\end{enumerate}

\section{Conclusion}
\label{sec:conclusion}

We have presented the LEGOLAS framework for Bayesian optimization of \znsse{} semiconductor composition in dye-sensitized solar cells. Through Gaussian Process regression with Expected Improvement acquisition, LEGOLAS achieves:

\begin{itemize}[noitemsep]
    \item \textbf{Efficient discovery:} Optimal composition found in 10 measurements (vs. 30+ for random search)
    \item \textbf{High accuracy:} \SI{0.019}{\volt} RMSE in voltage prediction (sub-measurement-noise)
    \item \textbf{Robust performance:} 98\% success rate across 100 independent optimization runs
    \item \textbf{Physical insight:} Identifies \zns{} as optimal for maximizing \voc{} (\SI{0.855}{\volt})
\end{itemize}

The framework integrates:
\begin{itemize}[noitemsep]
    \item Vegard's law for rapid band gap estimation
    \item AFLOW database for first-principles validation
    \item Low-cost Raspberry Pi hardware for accessibility
    \item Modular Python architecture for extensibility
\end{itemize}

Future work will extend LEGOLAS to multi-objective optimization (simultaneously maximizing \voc, \jsc, and FF), experimental validation with fabricated DSSCs, and transfer learning across related semiconductor families. The open-source framework provides a template for ML-accelerated materials discovery in both research and educational contexts.

\section*{Acknowledgments}

We thank the AFLOW consortium for database access and computational resources. S.C. acknowledges support from the Johns Hopkins Materials Science and Engineering Department and guidance from Dr. Corey Oses.

\section*{Data Availability}

All code, data, and documentation are available at:
\begin{itemize}[noitemsep]
    \item GitHub: \url{https://github.com/samridhi-chordia/legolas-znsse}
\end{itemize}

\newpage
\begin{thebibliography}{99}

\bibitem{mockus1974bayesian}
Mockus, J. (1974). On Bayesian methods for seeking the extremum. In \textit{Optimization Techniques IFIP Technical Conference} (pp. 400--404). Springer.

\bibitem{pedregosa2011scikit}
Pedregosa, F., Varoquaux, G., Gramfort, A., Michel, V., Thirion, B., Grisel, O., ... \& Duchesnay, É. (2011). Scikit-learn: Machine learning in Python. \textit{Journal of Machine Learning Research}, 12, 2825--2830.

\bibitem{rose2017aflux}
Rose, F., Toher, C., Gossett, E., Oses, C., Buongiorno Nardelli, M., Fornari, M., \& Curtarolo, S. (2017). AFLUX: The LUX materials search API for the AFLOW data repositories. \textit{Computational Materials Science}, 137, 362--370.

\bibitem{curtarolo2012aflow}
Curtarolo, S., Setyawan, W., Hart, G. L., Jahnatek, M., Chepulskii, R. V., Taylor, R. H., ... \& Levy, O. (2012). AFLOW: An automatic framework for high-throughput materials discovery. \textit{Computational Materials Science}, 58, 218--226.

\bibitem{yang2016modeling}
Yang, K., Oses, C., \& Curtarolo, S. (2016). Modeling off-stoichiometry materials with a high-throughput ab-initio approach. \textit{Chemistry of Materials}, 28(18), 6484--6492.

\bibitem{emmerich2011computation}
Emmerich, M. T., Deutz, A. H., \& Klinkenberg, J. W. (2011). Hypervolume-based expected improvement: Monotonicity properties and exact computation. In \textit{2011 IEEE Congress of Evolutionary Computation} (pp. 2147--2154). IEEE.

\bibitem{nsf2025dssc}
NSF Center for Innovation in Solar Fuels, California Institute of Technology. (2025). Solar energy conversion: Making a dye-sensitized solar cell. Available at: \url{https://www.scribd.com/document/485182350/example-DSSC-with-berry-1}

\bibitem{saar2023legolas}
Saar, L. (2023). Low-cost educational guided optimization lab for autonomous science (LEGOLAS). \textit{MRS Bulletin}, 48, 156--161. \url{https://link.springer.com/article/10.1557/s43577-022-00430-2}

\end{thebibliography}

\end{document}
